\documentclass[12pt,a4paper]{article}

% Packages
\usepackage[margin=1in]{geometry}
\usepackage{amsmath, amssymb, amsthm}
\usepackage{graphicx}
\usepackage{physics}
\usepackage{bm}
\usepackage{hyperref}
\usepackage{titlesec}

% Formatting
\titleformat{\section}{\large\bfseries}{\thesection}{1em}{}
\titleformat{\subsection}{\normalsize\bfseries}{\thesubsection}{1em}{}

% Title
\title{\textbf{Cauchy’s Stress Principle and Equilibrium Equations}}
\author{Prateeksha Sharma}
\date{}

\begin{document}

\maketitle

\begin{abstract}
Cauchy’s Stress Principle is a concept in continuum mechanics which states that the state of stress at a point within a material can be completely described by a second-order tensor $\bm{\sigma}$, which depends linearly on the unit vector $\mathbf{n}$. This article defines the principle, derives the equilibrium equations, and explains their physical meaning.
\end{abstract}

\section{Introduction}
In continuum mechanics, stress is the internal force transmitted within a material. While forces act on surfaces, the Cauchy stress tensor defines their intensity and direction at a point. Understanding this principle allows us to connect the physical idea of internal forces with a rigorous mathematical framework.

\section{Cauchy’s Stress Principle}
Consider an arbitrary volume element within a deformable body. The internal forces acting across any surface with unit normal vector $\mathbf{n}$ are described by the \textit{traction vector} $\mathbf{t}(\mathbf{n})$. Cauchy’s stress principle states:
\begin{equation}
    \mathbf{t}(\mathbf{n}) = \bm{\sigma} \, \mathbf{n}
\end{equation}
where $\bm{\sigma}$ is the \textit{Cauchy stress tensor}, a $3\times 3$ second-order tensor whose components $\sigma_{ij}$ represent the force in the $i$-direction acting on a face with outward normal in the $j$-direction.

\subsection{Physical Meaning}
The traction vector $\mathbf{t}(\mathbf{n})$ depends linearly on $\mathbf{n}$, and the stress tensor encapsulates all possible tractions on all possible orientations through a single point.

\section{Symmetry of the Stress Tensor}
By conservation of angular momentum for an infinitesimal element, it can be shown that:
\begin{equation}
    \sigma_{ij} = \sigma_{ji}
\end{equation}
This reduces the number of independent components from $9$ to $6$.

\section{Derivation of the Equilibrium Equations}
Consider a differential cube of size $\Delta x \times \Delta y \times \Delta z$ with density $\rho$ and subjected to body forces $\mathbf{b}$ (force per unit mass). Applying Newton’s second law in the absence of acceleration (static equilibrium), we have:
\begin{equation}
    \frac{\partial \sigma_{ij}}{\partial x_j} + \rho b_i = 0
\end{equation}
where $i = 1,2,3$ and Einstein summation convention is used over repeated indices.  
These are the \textit{static equilibrium equations}.

\section{Example: 1D Rod in Tension}
For a prismatic bar under axial load $P$, the only non-zero stress component is $\sigma_{xx}$, which is constant along the length if the load is uniformly distributed and the bar is in static equilibrium.

\section{Applications}
\begin{itemize}
    \item Stress analysis in structural members.
    \item Basis for finite element formulation in solid mechanics.
    \item Derivation of more advanced balance laws in continuum mechanics.
\end{itemize}

\section{Conclusion}
Cauchy’s Stress Principle provides the bridge between physical forces acting inside a body and the mathematics needed to describe them. The equilibrium equations play a central role, forming the basis for both analytical and numerical approaches in solving mechanics problems.

\section*{References}
\begin{thebibliography}{9}

\bibitem{fish2007}
Fish, J., \& Belytschko, T. (2007). \textit{A First Course in Finite Elements}. John Wiley \& Sons.

\bibitem{hjelmstad2005}
Hjelmstad, K. D. (2005). \textit{Fundamentals of Structural Mechanics} (2nd ed.). Springer.

\end{thebibliography}

\end{document}
